\documentclass[11pt,a4paper]{report}
\usepackage[utf8]{inputenc}
\usepackage{amsmath}
\usepackage{graphicx}
\usepackage{gensymb}
\usepackage{tikz}
\usepackage{pgfplots}
\usepackage{mathtools}
\usepackage{minted}
\usetikzlibrary{arrows,positioning}
\tikzset{
  %Define standard arrow tip
  >=stealth',
  % Define arrow style
  pil/.style={
      ->,
      thick,
      shorten <=2pt,
      shorten >=2pt,}
}
\usepackage{geometry}
\geometry{
  left=2cm,
  right=0.64cm,
  top=0.64cm,
  bottom=2cm
}
\usepackage{multicol}
\setlength{\columnsep}{1cm}
\graphicspath{ {images/} }

\begin{document}

\chapter{Semester 2 Examination 2013-2014\\CZ4013 Distributed System}

\begin{multicols*}{2}

\section{Question 1}

\noindent \textbf{Question 1a}

\noindent In at-least-once invocation semantics, if a client receives a reply message, the request has been executed for one or more than one times at the server. In at-most-once invocation semantics, if a client receives a reply message, the request has been executed for exactly one time. 

\noindent In at-least-once invocation semantics, when server receive a duplicate request, it will re-execute the request. In at-most-once invocation semantics, when server receive a duplicate request, it will send the previous reply again to the client.\\

\noindent \textbf{Question 1b}

\noindent Information given in the question:
\begin{itemize}
  \item Time spent in critical section by any process: $c$
  \item Message transmission delay: $d$
  \item Number of processes: $n$
\end{itemize}

\noindent For any algorithm, maximum throughput happens when when every process constantly wants to enter a critical section.\\

\noindent For central server algorithm, the throughput is:
$$\text{throughput}=\frac{1}{2d + c}$$

\noindent For ring-based algorithm, the throughput is:
$$\text{throughput}=\frac{1}{d + c}$$

\noindent For Ricart-and-Agrawala algorithm, the throughput is:
$$\text{throughput}=\frac{1}{d + c}$$

\noindent \textbf{Question 1c i}

\noindent When the freshness interval is 3 seconds, the client will only read the local cache within 3 seconds after the last validation time. If current time has passed 3 second freshness interval, the client will validation the file again with the server.\\

\noindent Take note that NFS server operations are stateless and idempotent.\\

\noindent Here is a list of consequences of client $A$ operations:
\begin{enumerate}
  \item $r_1$: Since the cache is empty, client contacts server to get latest file. Client read the local copy, which is \emph{not} the most recent update by $B$.
  \item $r_2$: Client reads local copy of the file, which is \emph{not} the most recent update by $B$.
  \item $r_3$: Since the file is not fresh, client contacts server to validate the file. Since the file has been updated by $B$, server transfer the latest file to client. Client then read the most recent update of the file. 
  \item $r_4$: Client read the local copy, which is \emph{not} the most recent update by $B$.
  \item $r_5$: Since the file is not fresh, client contacts server to validate the file. Since the file has been updated by $B$, server transfer the latest file to client. Client then read the most recent update of the file. 
  \item $r_6$: Client read the local copy, which is \emph{not} the most recent update by $B$.
  \item $r_7$: Since the file is not fresh, client contacts server to validate the file. Since the file has been updated by $B$, server transfer the latest file to client. Client then read the most recent update of the file. 
  \item $r_8$: Client read the local copy, which is the most recent update by $B$.
\end{enumerate}

\noindent The time points when $A$ needs to contact the server are $r_1,r_3,r_5,r_7$.\\

\noindent The read operations of $A$ that return the most recent update by $B$ are $r_3,r_5,r_7,r_8$. \\

\noindent \textbf{Question 1c ii}

\noindent Callback mechanism is used to maintain cache con- sistency of client side. However, if there is an update at server side before a client close a session / file, the callback has no effect on the file at currect session. \\

\noindent As a result, only the read operation at $r_6,r_7,r_8$ read the most recent update by $B$. \\

\noindent The time points when $A$ needs to contact the server are $r_1,r_5,r_6,r_7,r_8$.\\

\section{Question 2}

\noindent \textbf{Question 2a}

\begin{minted}{Java}
// StockNotification.java
public interface StockNotification 
    implements Remote {
  public void register(Callback cbObject, 
      int percentage) 
      throws RemoteException;
  public void deregister(Callback cbObject) 
      throws RemoteException;
}

// Callback.java
public interface Callback 
    implement Remote {
  public void stockAlert(String companyName) 
      throws RemoteException;
}
\end{minted}

\noindent \textbf{Question 2b i}

\noindent All possibility of $c$ and $c$:

\begin{center}
\begin{tabular}{|c|c|}
  \hline
  $c$ & $d$ \\
  \hline
  0  & 0  \\
  0  & 2  \\
  0  & 12 \\
  2  & 2  \\
  2  & 12 \\
  12 & 12 \\ \hline
\end{tabular}
\end{center}

\noindent \textbf{Question 2b ii}

\noindent Monotonic-read: If a client reads the value of a data object x, any successive read operation on x by that client will always return that same value or a more recent value.\\

\noindent There is no possibility that $c=10$. Let's assume that the question is wrong and $c=12$ instead, then the value of $d$ must be equal to $12$.\\

\noindent \textbf{Question 2b iii}

\noindent Read-your-write: The effect of a write operation by a client on a data object $x$ will always be seen by a successive read operation on $x$ by the same client.\\

\noindent $a$ must be equal to 2 and $b$ must be equal to 12.\\


\noindent \textbf{Question 2b iv}

\noindent Write-follow-reads: A write operation by a client on a data object $x$ following a previous read operation on $x$ by the same client is guaranteed to take place on the same or a more recent value of $x$ that was read.\\

\noindent $d$ can be 0, 2, or, 12.


\end{multicols*}
\end{document}
