\documentclass[11pt,a4paper]{report}
\usepackage[utf8]{inputenc}
\usepackage{amsmath}
\usepackage{graphicx}
\usepackage{gensymb}
\usepackage{tikz}
\usepackage{pgfplots}
\usepackage{mathtools}
\usepackage{minted}
\usetikzlibrary{arrows,positioning}
\tikzset{
  %Define standard arrow tip
  >=stealth',
  % Define arrow style
  pil/.style={
      ->,
      thick,
      shorten <=2pt,
      shorten >=2pt,}
}
\usepackage{geometry}
\geometry{
  left=2cm,
  right=0.64cm,
  top=0.64cm,
  bottom=2cm
}
\usepackage{multicol}
\setlength{\columnsep}{1cm}
\graphicspath{ {images/} }

\begin{document}

\chapter{Semester 2 Examination 2014-2015\\CZ4013 Distributed System}

\section{Question 1}

\noindent \textbf{Question 1a}
\begin{minted}{Java}
// Magazine.java
public class Magazine {
  public int issueNumber;
  public String title;
}

// MagazineNotification.java
public interface MagazineNotification implement Remote {
  void register(Callback cbObject) throws RemoteException;
  void deregister(Callback cbObject) throws RemoteException;
}

// ClientCallback.java
public interface ClientCallback implement Remote {
  void newMagazine(Magazine magazine) throws RemoteException;
}
\end{minted}

\begin{multicols*}{2}
\noindent \textbf{Question 1b i}

\noindent The $i$-th entry in the finger table contains the first node that succeeds $n$ by at least $2^{i-1}$ on the identifier circle.

\begin{center}
\begin{tabular}{|c|c|c|}
  \hline
  Entry & At-least & Identifier \\ \hline
  1     & 42 + 1   & N45        \\
  2     & 42 + 2   & N45        \\
  3     & 42 + 4   & N48        \\
  4     & 42 + 8   & N62        \\
  5     & 42 + 16  & N62        \\
  6     & 42 + 32  & N12        \\ \hline
\end{tabular}
\end{center}

\noindent \textbf{Question 1b ii}

\noindent In routing, each node $n$ sends a query for key $k$ to the node in entry $\lfloor log_2(k-n) \rfloor + 1$. To route from N42 to $K27 \equiv 27 + 63 = K90$, we use the

$$\lfloor log_2(90-42) \rfloor + 1 = 6\text{-th entry}$$

\noindent Now, we are at N12. To route from N12 to K27, we use the

$$\lfloor log_2(27-12) \rfloor + 1 = 4\text{-th entry}$$

\noindent The 4th entry of N12 is:

$$12 + 2^{4-1} = 22 \equiv \text{N22}$$

\noindent Now, we are at N22. To route from N22 to K27, we use the

$$\lfloor log_2(27-22) \rfloor + 1 = 3\text{-th entry}$$

\noindent The 3rd entry of N22 is:

$$22 + 2^{3-1} = 26 \equiv \text{N36}$$

\noindent Now, we are at N36, which contains the information of K27. \\

\noindent In summary, the route from N42 to K27 is N42 $\rightarrow$ N12 $\rightarrow$ N22 $\rightarrow$ N36.\\

\noindent \textbf{Question 1b iii}

\noindent For N36, any node added between N36 to N12 will cause it to update its finger table. For N42, any node added between N42 to N12 will cause it to update its finger table. So for both nodes to update their finger table, the new node must have an identifier between $42<n\le 63$ or $0\le n <12$.

\section{Question 2}

\noindent \textbf{Question 2a i}

\noindent Here is a list of all information:
\begin{itemize}
  \item Client computes arguments for a request: 2ms
  \item Server processes a request: 16ms
  \item Local OS send and receive operation: 1ms
  \item Network propagation for request and reply: 5ms
  \item Marshalling and unmarshalling: 1ms
\end{itemize}

\noindent Sequence of activities to complete a remote method invocation:
\begin{enumerate}
  \item Client compute arguments: 2ms
  \item Client marshalling: 1ms
  \item Client OS send: 1ms
  \item Network propagation: 5ms
  \item Server OS receive: 1ms
  \item Server unmarshalling: 1ms
  \item Server process: 16ms
  \item Server marshalling: 1ms
  \item Server OS send: 1ms
  \item Network propagation: 5ms
  \item Client OS receive: 1ms
  \item Client OS unmarshalling: 1ms
\end{enumerate}

\noindent The total time required to complete a remote method invocation is 36ms.\\

\noindent \textbf{Question 2a ii}

\noindent When the client is running on single-threaded, two requests are run sequentially, so $36 \times 2 =72$ms is required to return from two remote method invocation requests.\\

\noindent When the client is running on two threads, two requests are sent to the server at the same time, but server only can handle one request at a time. So the total time required to complete two requests are 56ms, because:
\begin{enumerate}
  \item Thread-1 and Thread-2 send the requests to server using 4ms
  \item Network propagation: 5ms
  \item Server processes the request from Thread-1 using 20ms
  \item Then, server processes the request from Thread-2 using 20ms
  \item Network propagation: 5ms
  \item Thread-2 processes the request using 2ms
\end{enumerate}

\noindent \textbf{Question 2b i}

\noindent When the freshness interval is 3 seconds, the client will only read the local cache within 3 seconds after the last validation time. If current time has passed 3 second freshness interval, the client will validation the file again with the server.\\

\noindent Take note that NFS server operations are stateless and idempotent.\\

\noindent Here is a list of consequences of client $A$ operations:
\begin{enumerate}
  \item $r_1$: server to transfer the file to $A$ because there is no local cache in $A$.
  \item $r_2$: $A$ read the local cache, and it is \emph{not} the most recent update by $B$
  \item $r_3$: $A$ validates the last modified time of the file with server, and found out that the file has been modified. So, the server transfer the file to $A$. $A$ reads the most recent update by $B$.
  \item $r_4$: $A$ read the local cache, which is the most recent update by $B$
  \item $r_5$: $A$ validates the last modified time fo the file with server, and found out that the file has not been modified. $A$ read local cache, which is the most recent update by $B$.
  \item $r_6$: $A$ read the local cache, which is \emph{not} the most recent update by $B$.
  \item $r_7$: $A$ validates the last modified time of the file with server, and found out that the file has been modified. So, the server transfer the file to $A$. $A$ reads the most recent update by $B$.
  \item $r_8$: $A$ read the local cache, which is the most recent update by $B$
\end{enumerate}

\noindent The read operation of $A$ that return the most recent update by $B$ are $r_3,r_4,r_5,r_7,r_8$\\

\noindent Transferring of files from server to client happens at $r_1,r_3,r_7$. The number of times the file is transferred from the server to $A$ is 3.\\

\noindent \textbf{Question 2b ii}

\noindent Callback mechanism is used to maintain cache con- sistency of client side. However, if there is an update at server side before a client close a session / file, the callback has no effect on the file at currect session. \\

\noindent As a result, only the read operation at $r_7,r_8$ read the most recent update by $B$. \\

\noindent The server sends the file and a callback promise to $A$ in $r_1$. Server has sent a callback to $A$ at $u_1$ to set the callback promise to \verb|cancelled|. At $r_7$, since the callback is \verb|cancelled|, the server sends the file to $A$ again. The number of times the file is transferred from the server to $A$ is 2. 

\section{Question 3}

\noindent \textbf{Question 3a i}

\noindent Timestamp $T_1,T_3,T_4$ are required because we need to calculate the round-trip-time and know local time of $S$ when sending message $m_2$.\\

\noindent \textbf{Question 3a ii}

\noindent The round-trip-time is:
$$RTT = T_4 - T-1 = 00:00:05.895$$

\noindent When $C$ receives message $m_2$ from $S$, the local time of $S$ is between:

$$[T_3, T_3+RTT]$$
$$[06:25:03.009, 06:25:08.904]$$

\noindent So $C$ should set its time to:
$$06:25:03.009, 2.947 = 06:25:05.956$$

\noindent \textbf{Question 3a iii}

\noindent The accuracy is bound within $\pm 00:00:02.947$ seconds.\\

\noindent \textbf{Question 3b i}

\noindent All timestamp $T_1,T_2,T_3,T_4$ are required because we need to consider the total transmission time and processing time of server $S$.\\

\noindent \textbf{Question 3b ii}

\noindent The total transmission time is:

$$(T_4 - T_1) - (T_3 - T_2) = 00:00:01.166$$

\noindent When $C$ receives $m_2$ from $S$, the time at $S$ is between: 
$$[T_3, T_3 + 00:00:01.166]$$
$$[06:25:03.009, 06:25:04.175]$$

\noindent So $C$ should set its time to:
$$T_3 + 0.583 = 06:25:03.592$$

\noindent \textbf{Question 3b iii}

\noindent The accuracy is bound within $\pm 00:00:00.583$ seconds.\\

\noindent \textbf{Question 3c}

\noindent The accuracy of Christian's algorithm or NTP depend on network delay. If the network delay is longer, the accuracy is worsen. To handle this problem in Christian's algorithm and NTP, we repeat the algorithm for reasonable amount of times at take the one with the highest accuracy.\\

\noindent \textbf{Question 3d}

\noindent A vector clock for $N$ processes is an array of $N$ integers. Each process has its own vector clock. Let $V_{p_i}$ be the vector clock for process $p_i$.\\

\noindent When a process $p_a$ perform an action in its own process, the vector clock is incremented $V_{p_a}[a] += 1$.\\

\noindent When a $p_a$ sends a request to $p_b$, it sends its vector clock $V_{p_a}$ to be as well. When $p_b$ receives the request, $p_b$ sets its vector clock to $V_{p_b}[i] = max(V_{p_a}[i], V_{p_b}[i]), \text{for } i=1\ldots N$. Then, $p_b$ increment the vector clock $V_{p_b}[b] += 1$.

\end{multicols*}
\end{document}
